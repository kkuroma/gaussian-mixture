\documentclass{article}

\usepackage{titlesec}
\titleformat{\section}{\Large\bfseries}{\thesection.}{1em}{}[\vspace{-0.8em}\rule{0.7\textwidth}{1pt}]
\titleformat{\subsection}{\large\bfseries}{\centering\thesubsection}{1em}{}[]
\usepackage{hyperref} 
\usepackage[utf8]{inputenc}
\usepackage{amsmath}
\usepackage{amssymb}
\newcommand{\Mod}[1]{\ (\mathrm{mod}\ #1)}
\newcommand{\sign}[1]{\text{sign}\left(#1\right)}
\newcommand{\ra}{\rightarrow}
\newcommand{\R}{\mathbb{R}}
\newcommand{\Z}{\mathbb{Z}}
\newcommand{\N}{\mathbb{N}}
\newcommand{\E}{\mathbb{E}}
\newcommand{\p}{\mathbb{P}}
\newcommand{\dd}{\text{d}}
\newcommand{\normal}{\mathcal{N}}
\newcommand{\var}{\text{Var}}
\newcommand{\AND}{\text{ and }}
\newcommand{\tr}[1]{\text{Tr}(#1)}
\newcommand{\INT}[1]{\text{int}(#1)}
\newcommand{\EXT}[1]{\text{ext}(#1)}
\newcommand{\BD}[1]{\partial(#1)}
\newcommand{\innerprod}[2]{\langle #1,#2 \rangle}
\newcommand{\magnitude}[1]{\lvert\lvert#1\rvert\rvert}
\newcommand{\proj}[2]{\text{proj}_#1(#2)}
\DeclareMathOperator{\Hom}{Hom}% preferred
\newcommand{\cMatrix}[2]{\begin{pmatrix}\vdots&\vdots&\vdots&\dots&\vdots\\#1_1&#1_2&#1_3&\dots&#1_#2\\\vdots&\vdots&\vdots&\dots&\vdots\\\end{pmatrix}}
\newcommand{\cVector}[2]{\begin{pmatrix}#1_1\\#1_2\\\vdots\\#1_#2\\\end{pmatrix}}
\newcommand{\rMatrix}[2]{\begin{pmatrix}\dots&#1_1&\dots\\\dots&#1_2&\dots\\\dots&#1_3&\dots\\\vdots&\vdots&\vdots\\\dots&#1_#2&\dots\\\end{pmatrix}}
\newcommand{\NCR}[2]{\begin{pmatrix}#1\\#2\end{pmatrix}}
\usepackage{pgfplots}
\usepackage{tikz}
\usepackage{float}
\usepackage{esint}
\usepackage{amsthm}
\usepackage{geometry}
\geometry{a4paper,margin=1in}
\usepackage{enumitem}
\usepackage[most]{tcolorbox}
\usepackage{fancyhdr}
\pagestyle{fancy}

\title{Gaussian Mixture Model and Applications with Expectation-Maximization}
\author{KK Thuwajit ``KKuRoMa'' and Khalid Al-Raisi ``Standard Deviation Male''}
\begin{document}

\newtheorem{theorem}{Theorem}
\newtheorem{lemma}{Lemma}
\theoremstyle{definition}
\newtheorem{definition}{Definition}
\newtheorem{example}{Example}
\maketitle
\tableofcontents
\newpage


\fancyfoot[CO,CE]{Fancy footnote hehe}

\section*{Preface}
We aim to establish the mathematical derivations that will be use throughout the document here.
\begin{theorem}[MLE of Categorical Sampling]
    Given $y_1,y_2,\dots,y_n$ are sampled IID from $\{1,2,\dots,k\}$ such that $\p(y_i=j)=\phi_j$, $\sum_{j=1}^k=\phi_j$, the MLE for $\phi_i$ is given by
    \[
        \hat{\phi}_j = \frac{1}{n} \sum_{i=1}^n \mathbb{I}\{y_i=j\}
    \]
\end{theorem}
\begin{proof}
    The likelihood of the parameters $\phi_1,\phi_2,\dots,\phi_k$ given observations $y_1,y_2,\dots,y_n$ can be expressed as
    \[
    f(\phi_1,\phi_2,\dots,\phi_k) = \prod_{i=1}^n \phi_{y_i} = \prod_{j=1}^k \phi_{j} ^ {\frac{1}{n} \sum_{i=1}^n \mathbb{I}\{y_i=j\}}
    \]
    and consequently the log-likelihood
    \[
    g(\phi_1,\phi_2,\dots,\phi_k) = \frac{1}{n} \sum_{j=1}^k \log(\phi_{j}) \left(\sum_{i=1}^n \mathbb{I}\{y_i=j\}\right)
    \]
    we are able to solve for the MLE using Lagrangian multipliers. Consider the objective function 
    \[
    \mathcal{L}(\phi_1,\phi_2,\dots,\phi_k,\lambda) = \sum_{j=1}^k \log(\phi_{j}) \left(\sum_{i=1}^n \mathbb{I}\{y_i=j\}\right) + \lambda\left(1-\sum_{j=1}^k\phi_j\right)
    \]
    Taking the partial derivative for the gradient
    \[
    \frac{\partial}{\partial \phi_l} \mathcal{L}(\phi_1,\phi_2,\dots,\phi_k,\lambda) = \frac{\sum_{i=1}^n \mathbb{I}\{y_i=l\}}{\phi_l} - \lambda
    \]
    and the second partial derivative for the hessian
    \[
        \frac{\partial^2}{\partial \phi_l^2} \mathcal{L}(\phi_1,\phi_2,\dots,\phi_k,\lambda) = -\frac{\sum_{i=1}^n \mathbb{I}\{y_i=l\}}{\phi_l^2}
        \text{ and }
        \frac{\partial}{\partial \phi_l}\frac{\partial}{\partial \phi_m}\mathcal{L}(\phi_1,\phi_2,\dots,\phi_k,\lambda) = 0
    \]
    We can see that the hessian is negative semi-definite, meaning setting the gradient as $\mathbf{0}$ or equivalently $\phi_l=\frac{\sum_{i=1}^n \mathbb{I}\{y_i=l\}}{\lambda}$ yields the maximum. Since
    \[
        1 = \sum_{j=1}^k\phi_j = \sum_{j=1}^k \frac{\sum_{i=1}^n \mathbb{I}\{y_i=j\}}{\lambda} = \frac{n}{\lambda}
    \]
    therefore, $\lambda=n$. Put together, the empirical prediction $\hat{\phi_j} = \frac{\sum_{i=1}^n \mathbb{I}\{y_i=j\}}{n}$ is the MLE, completing the proof
\end{proof}

\begin{theorem}[MLE of Multivariate Gaussian]
    Given $\mathbf{x}_1,\mathbf{x}_2,\dots,\mathbf{x}_n\in\R^d$ are sampled IID from $\normal(\boldsymbol{\mu},\Sigma)$, the MLE for $\boldsymbol{\mu}$ and $\Sigma$ are given by
    \[ 
        \boldsymbol{\mu} = \frac{1}{n} \sum_{i=1}^n \mathbf{x}_i \text{ and } \Sigma = \frac{1}{n} \sum_{i=1}^n \mathbf{x}_i \mathbf{x}_i^\top
    \]
\end{theorem}
\begin{proof}
    The likelihood of the parameters $\boldsymbol{\mu}$ and $\Sigma$ given observations $\mathbf{x}_1,\mathbf{x}_2,\dots,\mathbf{x}_n$ can be expressed as
    \[
    f(\boldsymbol{\mu},\Sigma) = \prod_{i=1}^n \frac{1}{(2\pi)^{d/2}|\Sigma|^{1/2}} \exp\left( -\frac{1}{2} (\mathbf{x}_i - \boldsymbol{\mu})^\top \Sigma^{-1} (\mathbf{x}_i - \boldsymbol{\mu}) \right)
    \]
    and consequently the log-likelihood
    \[
    g(\boldsymbol{\mu},\Sigma) = -\frac{n}{2} \log |\Sigma| - \frac{nd}{2} \log(2\pi) - \frac{1}{2} \sum_{i=1}^n (\mathbf{x}_i - \boldsymbol{\mu})^\top \Sigma^{-1} (\mathbf{x}_i - \boldsymbol{\mu})
    \]
    To find the MLE, we maximize $g(\boldsymbol{\mu},\Sigma)$ with respect to $\boldsymbol{\mu}$ and $\Sigma$. First, optimizing with respect to $\boldsymbol{\mu}$, we take the gradient and hessian
    \[
    \frac{\partial}{\partial \boldsymbol{\mu}} g(\boldsymbol{\mu},\Sigma) = \Sigma^{-1} \sum_{i=1}^n (\mathbf{x}_i - \boldsymbol{\mu})
    \text{ and }
    \frac{\partial^2}{\partial \boldsymbol{\mu}^2} g(\boldsymbol{\mu},\Sigma) = -\Sigma^{-1}
    \]
    Since $\Sigma$ has to be positive semi-definite to be a covariance, so does $\Sigma^{-1}$, which implies the gradient to zero or
    \[
    \Sigma^{-1} \sum_{i=1}^n (\mathbf{x}_i - \boldsymbol{\mu}) = 0
    \implies
    \sum_{i=1}^n (\mathbf{x}_i - \boldsymbol{\mu}) = 0
    \implies
    n\boldsymbol{\mu} = \sum_{i=1}^n \mathbf{x}_i
    \implies
    \boldsymbol{\mu} = \frac{1}{n} \sum_{i=1}^n \mathbf{x}_i
    \]
    yields the maximum. Next, optimizing with respect to $\Sigma$, plugging $\hat{\boldsymbol{\mu}}$ back in, the log-likelihood simplifies to
    \[
    g(\boldsymbol{\mu},\Sigma) = -\frac{n}{2} \log |\Sigma| - \frac{1}{2} \sum_{i=1}^n (\mathbf{x}_i - \hat{\boldsymbol{\mu}})^\top \Sigma^{-1} (\mathbf{x}_i - \hat{\boldsymbol{\mu}})
    \]
    Here, we use the trace trick:
    \[
    -\frac{n}{2} \log |\Sigma| - \frac{1}{2} \sum_{i=1}^n \text{Tr}\left((\mathbf{x}_i - \hat{\boldsymbol{\mu}})^\top \Sigma^{-1} (\mathbf{x}_i - \hat{\boldsymbol{\mu}})\right)
    \]
    Taking the derivative with respect to $\Sigma^{-1}$ using the these identities
    \[
    \frac{\partial}{\partial \Sigma^{-1}} \log |\Sigma| = \Sigma
    \quad \text{and} \quad
    \frac{\partial}{\partial \Sigma^{-1}} \operatorname{tr}(A\Sigma^{-1}) = -A
    \]
    we obtain
    \[
    \frac{\partial}{\partial \Sigma^{-1}} g(\boldsymbol{\mu},\Sigma) = \frac{n}{2}\Sigma - \frac{1}{2}\sum_{i=1}^n (\mathbf{x}_i - \hat{\boldsymbol{\mu}})(\mathbf{x}_i - \hat{\boldsymbol{\mu}})^\top
    \]
    Setting the gradient to zero, we get
    \[
    n\Sigma = \sum_{i=1}^n (\mathbf{x}_i - \hat{\boldsymbol{\mu}})(\mathbf{x}_i - \hat{\boldsymbol{\mu}})^\top
    \implies
    \Sigma = \frac{1}{n} \sum_{i=1}^n (\mathbf{x}_i - \hat{\boldsymbol{\mu}})(\mathbf{x}_i - \hat{\boldsymbol{\mu}})^\top
    \]
    Put together, the empirical predictions are indeed the MLE, thus completing the proof.
\end{proof}
\section{Gaussian Mixture Model}
\subsection{Defining Gaussian Mixture}
\begin{definition}[Gaussian Mixture Problem]
    For $j=1,2,\dots,k$ suppose there exists some fixed, hidden parameters $\phi_j \in \R,\boldsymbol{\mu}_j \in \R^d,\Sigma_j \in \R^{d\times d}$ such that $\sum_{j=1}^k \phi_j = 1$ and $\Sigma_j$ is a valid covariance matrix, we consider the following procedure
    \begin{itemize}
        \item For $i=1,2,\dots,n$ sample IID $y_i$ from $\{1,2,\dots,k\}$ such that $\p(y_i)=\phi_j$
        \item For $i=1,2,\dots,n$ sample IID $\textbf{x}_i$ from $\normal(\boldsymbol{\mu}_{y_i},\Sigma_{y_i})$
    \end{itemize}
    There are two versions of the problems:
    \begin{itemize}
        \item \textbf{Supervised}: for $i=1,2,\dots,n$ we are given $\textbf{x}_i,y_i$. Find the estimate for $\phi_j,\boldsymbol{\mu}_j,\Sigma_j$
        \item \textbf{Unsupervised}: for $i=1,2,\dots,n$ we are given only $\textbf{x}_i$. Find the estimate for $\phi_j,\boldsymbol{\mu}_j,\Sigma_j$
    \end{itemize}
\end{definition}
This is assuming that there exists $k$ different Gaussian clusters, and $\mathbf{x}_i$ is sampled from one of them with probability specified by $\phi_j$. Notice that the supervised version of the problem essentially boils down to running the regular Multivariate Gaussian problem for each class $y_i$ can take. We are more interested in the unsupervised problem, as the MLE could not be solved in a closed form.
\subsection{Potential Applications of Gaussian Mixture}
We present two applications of the unsupervised version of the Gaussian Mixture defined above. The exact implementations will be explained in later sessions.
\begin{itemize}
    \item \textbf{Unsupervised classification}: the resulting estimate for the Gaussian clusters' mean and covariance allows estimates for the unknown label $y_i$ to be retrieved from $\textbf{x}_i$.
    \item \textbf{Conditional generation}: the Gaussian Mixture model allows data outside of the given set to be sampled from the resulting Gaussian distributions, conditioned on one of the possible classes.
\end{itemize}
\subsection{Gaussian Mixture as MLE Problem}
The likelihood of the 
\section{Expectation-Maximization (EM) Algorithm}
\subsection{Defining EM Algorithms}
\subsection{EM Algorithm for Gaussian Mixture Models}
\subsection{EM Algorithm for Variational Inference}
\section{Applications of EM: Unsupervised Classification}
\begin{definition}[Unsupervised Classification]
    Given data vectors $\mathbf{x}_1,\mathbf{x}_2,\dots,\mathbf{x}_n\in\R^d$ and a set of possible classes $\mathcal{C}=\{1,2,\dots,k\}$, we attempt to find $z_1,z_2,\dots,z_n\in \mathcal{C}$ and parameters $\theta$ such that $\prod_{i=1}^n p(\mathbf{x}_i,z_i;\theta)$. In other words, the MLE under a known distribution type is maximized.
\end{definition}
This definition can seen as a reformulation of the ``Distribution Learning with Hidden Latent Variables'' where $\mathbf{z}_i$ is limited to a discrete set. By solving for $q$ with EM, we're able to derive the MLE for the classes $z_i$ as follows
\begin{align*}
    \hat{z} 
    &= \arg\max_{z\in\mathcal{C}} p(z|\mathbf{x};\theta) \\
    &= \arg\max_{z\in\mathcal{C}} \frac{p(\mathbf{x}|z;\theta)p(z)}{p(\mathbf{x})} & \text{Bayes' Rule}\\
    &= \arg\max_{z\in\mathcal{C}} p(\mathbf{x}|z;\theta)p(z) & p(\mathbf{x})\text{ is constant} \\
    &= \arg\max_{z\in\mathcal{C}} p(\mathbf{x}|z;\theta)\hat{q}(z) & \hat{q}\text{ is an estimate of the prior}
\end{align*}

\subsection{K-means Clustering}
The K-means clustering finds an estimate to the unsupervised classification problem using the following procedure. First assign centroids $\boldsymbol{\mu}_j^{(0)}$ for $j=\{1,2,\dots,k\}$ to random points. Until convergence, run
\begin{itemize}
    \item $\mathcal{S}_j^{(t)} = \{i\in\{1,2,\dots,n\}:j=\arg\max_l\|\boldsymbol{\mu}_l^{(t)}-\mathbf{x}_i\|_2^2\}$
    \item $\boldsymbol{\mu}_j^{(t+1)}=\frac{1}{|\mathcal{S}_j^{(t)}|}\sum_{i\in \mathcal{S}_j^{(t)}} \mathbf{x}_i$
\end{itemize}
Finally, assign $\hat{z}_i=\arg\min_j\|\boldsymbol{\mu}_j^{(t)}-\mathbf{x}_i\|_2^2$. We will first argue that the algorithm above can be seen as a special case of the EM algorithm. Consider a simpler case of the Gaussian Mixture problem such that:
\begin{itemize}
    \item For $j=\{1,2,\dots,k\}$, $\phi_j=\frac{1}{k}$ (each cluster has an equal chance of being assigned)
    \item For $j=\{1,2,\dots,k\}$, $\Sigma_j=\sigma^2\mathbf{I}_{d\times d}$, this value of $\sigma$ will be clarified later.
\end{itemize}
Knowing $\phi_j$, we are able to simply the E-step as follows:
\[
    q_i^{(t)}(z) = p(z|\mathbf{x}_i;\theta^{(t)}) = \frac{p(\mathbf{x}_i,z;\theta^{(t)})}{\sum_{j=1}^k p(\mathbf{x}_i,j;\theta^{(t)})} = \frac{\exp(-\|\mathbf{x}_i-\boldsymbol{\mu}^{(t)}_z\|_2^2/2\sigma^2)}{\sum_{j=1}^k \exp(-\|\mathbf{x}_i-\boldsymbol{\mu}_j\|_2^2/2\sigma^2)}
\]
the next trick involves taking the limit of $\sigma\to0$, which implies that each Gaussian cluster converges to an impulse function. This is also called \textbf{hard-labeling}. We could simply to the final term as $1/\sigma$ dominates in the term where $\|\mathbf{x}_i-\boldsymbol{\mu}^{(t)}_j\|_2^2$ is smallest.
\[
\lim_{\sigma\rightarrow0} q_i^{(t)}(z) = \lim_{\sigma\rightarrow0} \frac{\exp(-\|\mathbf{x}_i-\boldsymbol{\mu}^{(t)}_z\|_2^2/2\sigma^2)}{\sum_{j=1}^k \exp(-\|\mathbf{x}_i-\boldsymbol{\mu}^{(t)}_j\|_2^2/2\sigma^2)} = \begin{cases}
    1 & z = \arg\min_j\|\boldsymbol{\mu}_j^{(t)}-\mathbf{x}_i\|_2^2 \\
    0 & \text{otherwise}
\end{cases}
\]
substituting to the M-step $\boldsymbol{\mu}_z^{(t)} = \frac{\sum_{i=1}^n q_i^{(t)}(z) \mathbf{x}_i}{\sum_{i=1}^n q_i^{(t)}(z)}$: defining $\mathcal{S}_j^{(t)} = \{i\in\{1,2,\dots,n\}:j=\arg\max_l\|\boldsymbol{\mu}_l^{(t)}-\mathbf{x}_i\|_2^2\}$ like above,
\[
    \boldsymbol{\mu}_z^{(t)} = \frac{1}{|\mathcal{S}_j^{(t)}|}\sum_{i\in \mathcal{S}_j^{(t)}} \mathbf{x}_i
\]
which agrees with expression presented in the K-means Clustering algorithm. As such, as a direct result of the EM Algorithm Convergence theorem, the K-means Clustering guarantees that the log-likelihood decreases monotonically in time.

\subsection{Gaussian Mixture Classification}
As expressed in the beginning of this section, we are able to express the MLE for the label as such:
\begin{align*}
    \hat{z}_i
    &= \arg\max_z p(\mathbf{x}_i|z;\theta)\hat{q}(z) \\
    &= \arg\max_z \frac{\phi_z}{|\Sigma_z|^{1/2}} \exp\left(\frac{\|\mathbf{x}_i-\boldsymbol{\mu}_z\|^\top\Sigma_z^{-1}\|\mathbf{x}_i-\boldsymbol{\mu}_z\|}{2}\right)
\end{align*}
\subsection{Comparison}
The Gaussian Mixture method generalizes K-means Clustering with the following distinction
\begin{itemize}
    \item \textbf{Soft-labeling}: the Gaussian Mixture model considers the probability of each cluster having distinct spread, and that all points have a possibility of being a tail outlier from a faraway cluster.
    \item \textbf{Cluster probability}: as a direct consequently of hard-labeling, the K-means Clustering algorithm is unable to produce a distribution-type posterior probability $p(z|\mathbf{x}_i;\theta^{(t)})$, while the Gaussian Mixture method naturally produces this value.
\end{itemize}
Still, both model both utilizes the Gaussian Distribution to model the associated cluster and apply the EM Algorithm to solve for an MLE estimate. As such, both models suffer greatly from poorly initialized $\boldsymbol{\mu}_j$ centroids as their convergence only guarantees up to the monotonic decrease of the loss function, leading to local minima situations.
\section{Conditional Generation}
\subsection{Pure Gaussian Mixture Models}
\subsection{Gaussian Mixture Variational Autoencoders}
\subsection{Comparison}

\end{document}
